\begin{block}{Introduction}
We study the syntactic-semantic dependency correlation through the mutual information gain of SSDP hop patterns.
% \setlist{nolistsep}
\begin{itemize}
    % \item Syntactic dependencies: Head$\rightarrow$Dependent relations.
    % \item Semantic dependencies: Predicate$\rightarrow$Argument relations. We denote the relation type as semantic labels.
    \vspace{-0.8cm}
    \item Semantic label: relation type (e.g., A0) of semantic dependencies
    \item SSDP: Shortest Syntactic Dependency Path connecting predicate-argument pairs
    \item Hop pattern: the transition pattern of SSDP
\end{itemize}
\begin{figure}
    \centering
    \captionsetup{justification=centering}
    \includegraphics[width=0.7\textwidth]{images/syn-sem-dep-example.png}
    \caption{Example semantic dependencies, SSDP of the semantic dependencies, and hop patterns. Solid lines underline predicates and dash lines underline arguments.}
    \label{fig:syn-sem example}
\end{figure}
% Hop patterns distinguish the two semantic dependencies with the same predicate-argument pair.\\
% \vspace{0.5cm}
% \textbf{Hypothesis: Different hop patterns have unique distributions of semantic labels.}
Intuitions:
\begin{itemize}
\vspace{-1cm}
    \item Semantic dependencies are modeled as a distribution of the predicate and the argument \cite{dozat-manning-2018-simpler, strubell-etal-2018-linguistically, he-etal-2018-syntax}
    \item Semantic parsers are vulnerable to semantic dependencies (Figure \ref{fig:syn-sem example}) that have different semantic labels but share the same predicate and argument
    \item Hop patterns can distinguish between the two semantic dependencies
\end{itemize}

\textbf{Hypothesis: Different hop patterns have unique distributions of semantic labels}


\end{block}